\section{Amenable Groups}%
\label{sec:Amenable Groups}

\subsection{Finitely Additive Probability Spaces and Means}

Before studying Amenable groups we will need to build up some results from
probability theory first. Suppose that $E$ is a set.
\begin{defn}
  A finitely additive probability measure is a function $\mu: \mathcal{P}(E)
  \to [0, 1]$ that satisfies the following:
  \begin{enumerate}
    \item $\mu(E) = 0$ 
    \item If $A, B \subseteq E, A \cap B = \emptyset$ then $\mu(A \cup B)
      = \mu(A) + \mu(B)$
  \end{enumerate}
\end{defn}

\begin{propn}
  Suppose that $\mu$ is a finitely additive probability measure. Then the
  following statements are true.
  \begin{enumerate}
    \item $\mu(\emptyset) = 0$
    \item $\mu(A \cup B) = \mu(A) + \mu(B) - \mu(A \cap B)$
    \item $\mu(A \cup B) \le \mu(A) + \mu(B)$
    \item If $A \subseteq B$ then $\mu(B \ A) = \mu(B) - \mu(A)$
    \item If $A \subseteq B$ then $\mu(A) \le \mu(B)$
  \end{enumerate}
\end{propn}

Our next goal in the chapter is going to be to study 'means' of finitely
additive probability spaces.  To do so we are going to need to study the space
$\ell^{\infty}(E)$ otherwise the space of bounded continuous functions from $E
\to \mathbb{R}$. Notice that this is a normed space with the norm givin by
$\|\|_{\infty}$ defined as
\[
\|x\|_{\infty} = \sup_{s \in E}|x(s)|
.\] 
Moreover, we give this space the partial ordering $\le$ by
\[
x \le y \iff x(a) \le y(a) \forall a \in E
.\] 
We will also occasionally define $\lambda \in \ell^{\infty}$ by functions that
are identically $\lambda$ on all of $E$. Equipped with this we will define the
mean as

\begin{defn}
  A mean is a linear transformation $m: \ell^{\infty} \to \mathbb{R}$ that
  satisfies the following properties:
  \begin{enumerate}
    \item $\mu(1) = 1$
    \item For all $x \ge 0$, $\mu(x) \ge 0$
  \end{enumerate}
\end{defn}

\begin{ex}
  There is a very important example of a mean that we will be often considering
  going forward. Suppose that $S \subseteq E$ and $S$ is a countable subset
  additionally suppose that we have a function $f: S \to \mathbb{R}$ that
  satisfies the following properties:
  \begin{enumerate}
    \item $f(x) \ge 0$ for all $s \in S$
    \item $\sum_{s \in S}f(s) = 1 $
  \end{enumerate}
  Then we can define a mean $m_f$ by
  \[
  m_{f}(x) = \sum_{s \in S} f(s)x(s)
  .\] 
  Notice that this is a mean on $E$ and moreover this gives us
  a finiteness/countability condition that we can work with for arbitrary sets
  $E$. We can say that a mean $m$ on $E$ has countable/finite support if there
  exists a countable/finite set $S$ and a function $f: S \to \mathbb{R}$ that
  satisfies the previous conditions such that $m = m_f$.
\end{ex}

\begin{propn}
  Some properties of means are as follows:
  \begin{enumerate}
    \item $m(\lambda) = \lambda$
    \item $x \le y \implies m(x) \le m(y)$
    \item $\inf_E \le m(x) \le \sup_E$ 
    \item $|m(x)| \le \|x\|_{\infty}$
  \end{enumerate}
\end{propn}
\begin{proof}
\textbf{Mostly boring fill in later.}
\end{proof}

If we consider now the topological dual of $\ell^{\infty}$,
$(\ell^{\infty}(E))^{*}$ namely this is the space of all continuous functions $u:
\ell^{\infty} \to \mathbb{R}$ we note that this is a Banach space with the
operator norm $\|\cdot\|$
\[
\|u\| = sup_{\|x\|_{\infty} \le 1}|u(x)| 
.\] 

\begin{propn}
  Let $m: \ell^{\infty}(E) \to \mathbb{R}$ be a mean for $E$ then $m \in
  (\ell^{\infty})^{*}$ and $\|m\| = 1$
\end{propn}
\begin{proof}
  \textbf{Also mostly boring\ldots left as an exercise to future me.}
\end{proof}

Going forward we will denote $\mathcal{M}$ to be the set of means and
$\mathcal{PM}$ to be the set of finitely additive probability measures. Indeed
we will try and construct a bijection between the two sets. This will allow us
to talk about probability measures as elements in the dual space of
$\ell^{\infty}$ and allow us to use tools from functional analysis to help us.

Consider the characteristic map for some set $A$ $\chi_A: E \to X$, recall that 
this map is given by $\chi(a) = 1$ if $a \in A$, and $\chi(a) = 0$ otherwise.
Suppose that $m \in \mathcal{M}$, we define $\widehat{m}: \mathcal{P}(E) \to
\mathbb{R}$ given to us by
\[
\widehat{m}(A) = m(\chi_A)
.\] 
Notice that this is indeed a mean on $E$. \textbf{Exercise to future self is to
double check this}. We notice here that $\widehat{m} \in
\mathcal{P}\mathcal{M}(E)$.

\begin{thm}
  The map $\Phi: \mathcal{M}(E) \to \mathcal{P}\mathcal{M}(E)$ defined by
  $\Phi(m) = \widehat{m}$ is a bijection.
\end{thm}

Before we begin lets consider the set $\mathcal{E}(E)$ defined as
\[
  \mathcal{E}(E) = \{x: E \to \mathbb{R}: \textrm{ $x$ takes on finitely many
  values}\} 
.\] 

\begin{lemma}
  $\mathcal{E}(E)$ is a subspace of $\ell^{\infty}(E)$
\end{lemma}
\begin{proof}
  This is quick to see. Suppose that $x, y \in \mathcal{E}(E)$ and let $c \in
  \mathbb{R}$. Notice that both $x$ and $y$ take on finitely many values
  we notice that $cy$ also takes on finitely many values. Moreover, $x + cy$
  must also take on finitely many values therefore we have that
  \[
  x + cy \in \mathcal{E}(E)
  .\] 
  Additionally this is clearly a subset of $\ell^{\infty}(E)$. As a result, we
  have that $\mathcal{E}(E)$ is a subspace of $\ell^{\infty}(E)$ as needed.
\end{proof}

\begin{lemma}
  $\mathcal{E}(E)$ is dense in $\ell^{\infty}(E)$. 
\end{lemma}
\begin{proof}
  Let $x \in \ell^{\infty}(E)$ and let $\alpha = \inf_E x, \beta = \sup_E x$. 
  Suppose that $\epsilon > 0$ is given. We can choose $n$ such that 
  $\frac{\beta - \alpha}{n} < \epsilon$. Now let $\lambda_i = \alpha
  + i \frac{(\beta - \alpha)}{n}$ for $i = 1, \ldots,n$. Now we can define the
  function $y: E \to \mathbb{R}$ given by
  \[
  y(a) = \min \left\{ \lambda_i : x(a) \le \lambda_i \right\}
  \] 
  for all $a \in E$. Notice that $y \in \mathcal{E}(E)$ and we have that 
  \[
  \|x - y\| < \frac{\beta - \alpha}{n} < \epsilon
  \] 
  by definition. As a result, we have that $\mathcal{E}(E)$ is dense in
  $\ell^{\infty}(E)$ as needed.
\end{proof}

The above two lemmas are enough to show that the map $\Phi$ is injective.

\begin{proof}[Proof of injectivity of Theorem 4.1.7]
  Suppose that $\Phi(m_1) = \Phi(m_2)$ then we know that
  \[
  \hat{m_1} = m_1(\chi_A) = m_2(\chi_A) = \hat{m_2}
  .\] 
  Since we know that $m_1, m_2$ are continuous functions that agree on
  $\mathcal{E}(E)$. \textbf{Double check the net argument here}(follows from
  the density of $\mathcal{E}(E)$ and continuity of $M$. Since this is
  a Hausdorff space and $m_1, m_2$ are continuous. Therefore, we have that
  $\Phi$ is injective as needed.
\end{proof}

\subsection{Properties of Set Means}

In the previous section we considered the toplogy on $(\ell^{\infty}(E))^{*}$ given
to us by $\|\cdot\|$. This is what we call the strong-topoloy. For this section
we will consider the weak-* topology on $\ell^{\infty}(E)$. Recall that this is
the smallest topology for which the evaluation map is continuous. Namely that
\begin{align*}
  \psi_x : (\ell^{\infty}(E))^{*} &\to \mathbb{R} \\
                 u &\mapsto u(x)
.\end{align*}

It is also important to notice that $\mathcal{M}(E) \subseteq \mathcal{U}$
where $\mathcal{U}$ denotes the unit ball of $(\ell^{\infty}(E))^{*}$. Defined by
 \[
\mathcal{U} = \{u: \|u\| = 1\} \subseteq (\ell^{\infty}(E))^{*}
.\] 
Recall that by the Banach Alaoglu theorem $\mathcal{U}$ is a compact set. We
wish to show now that the set $\mathcal{M}(E)$ is a compact convex set with 
resepct to the weak-* topology. 

\begin{thm}
  $\mathcal{M}(E)$ is a compact convex subset of $((\ell^{\infty})(E))^{*}$ with
  respect to the weak-* topology.
\end{thm}
\begin{proof}
  First we will show that $\mathcal{M}(E)$ is a convex subset. To do this we
  will show that for $m_1, m_2 \in \mathcal{M}(E)$ the straight line $\gamma:
  \ell^{\infty}(E) \to \mathbb{R}$ given by $\gamma(x) = tm_1(x) + (1-t)m_2(x) \in 
  \mathcal{M}(E)$ for $t \in [0, 1]$. We first check that $\gamma(1) = 1$. 
  Here we clearly have that
  \[
  \gamma(1) = tm_1(1) + (1 - t)m_2(1) = t + 1 - t = 1
  .\] 
  Additionally, we notice that for all $x \in \ell^{\infty}(E)$ we have that
  \[
  \gamma(x) = tm_1(x) + (1 - t)m_2(x) \ge 0
  .\] 
  Therefore we can conclude that $\gamma \in \mathcal{M}(E)$ and thus we see
  that $\mathcal{M}(E)$ is convex.\\
  Next to show that $\mathcal{M}(E)$ is (weakly)-compact, we note that it is 
  sufficient to show that it is closed in the weak-* topology. This is because
  we know that $\mathcal{M}(E) \subseteq \mathcal{U}$ and $\mathcal{U}$ is
  compact. Suppose that $(m_i)_{i \in I}, x_i \in \mathcal{M}(E)$ is a net 
  that converges to a point $(u \in \ell^{\infty})^{*}$. Here we notice that
  for every $i \in I$ $\psi_1(m_i) = 1$ and $\psi_x(m_i) \ge 0$. Now by the
  continuitiy of the evaluation map in the weak-* topology we see that 
  \[
  u(1) = \psi_1(u) = 1 \qquad \textrm{and} \qquad u(x) = \psi_x(u) \ge 0 
  \] 
  and so we see that $u \in \mathcal{M}(E)$. Therefore, since out net was
  arbitrary we see that every net of points in $\mathcal{M}(E)$ converges to
  a point in $\mathcal{M}(E)$ and therefore $\mathcal{M}(E)$ is closed in the
  weak-* topology. As needed.
\end{proof}

\subsection{Means and Measures on Groups}

Notice that in the previous sections our definitions of means and measures were
on sets in general. Here we will see that when the underlying set is a group
that we have some additional properties on top of the usual properties of
measures and means. Namely if $G$ is a group then there are natrual actions of
$G$ that can be put on $\mathcal{PM}(G)$ and on $\mathcal{M}(G)$. 

\begin{defn}[G acting on $\mathcal{PM}(G)$]
  We can define the left action of $G$ on $\mathcal{PM}(G)$ as the map $g\mu:
  \mathcal{P}(G) \to [0, 1]$ given by
  \[
  g\mu(A) = \mu(g^{-1}A)
  .\] 
  for all $A \in \mathcal{P}(G)$. We can do something similar for the right
  action by defining a function $\mu g: \mathcal{P} \to [0,1]$ given by
  \[
  \mu g(A) = \mu(A g^{-1})
  .\] 
\end{defn}

\begin{defn}[Right G-shift on $\mathbb{R}^{G}$ ]
 Now recall from the previous section we defined the shift action on the group
$A^{G}$. Here if we take $A = \mathbb{R}$ then we can define the g-shift map as
\[
  gx(g') = x(g^{-1} g') \textrm{ (For all $g' \in G$)}
.\] 
We can similarly define the element $xg$ as
\[
xg(g') = x(g'g^{-1})
.\] 
This gives us a right action of $G$ on $\mathbb{R}^{G}$.
\end{defn}
Notice now that by duality we get a left and a right action of $G$ on
$\ell^{\infty}(G)^{*}$. Namely that for $g \in G$ and $u \in
\ell^{\infty}(G)^{*}$ we can define the elements $gu$ and $ug$ as
\[
  gu(x) = u(g^{-1}x) \tag{Similar for $ug$ }
.\] 
Notice that that the set $\mathcal{M}(G)$ is invariant under both actions of
$G$ on $\ell^{\infty}(G)^{*}$.

\begin{propn}
  The left action of $G$ on $\mathcal{M}(G)$ is affine and continuous with
  respect to the weak-* topology on $\mathcal{M}(G)$.
\end{propn}
\begin{proof}
  It is clear that the left action of $G$ on $\ell^{\infty}(G)^{*}$ is linear.
  Moreover, in the weak-* topology these actions are also continuous since if
  we fix $g \in G$ then the map $u \mapsto gu$ is continuous as for each $x \in
  \ell^{\infty}(G)$ the map $u \mapsto gu(x)$ is the evaluation map at
  $g^{-1}(x)$. These are continuous by definition. Since $\mathcal{M}(G)$ is
  a convex subset of $\ell^{\infty}(G)^{*}$ we have that the restriction of the
  action to $\mathcal{M}(G)$ is continuous. As needed.
\end{proof}
For the left action we provide a similar argument.

For a map $\mu \in \mathcal{PM}(G)$ we have that map $\mu*: \mathcal{P}(G) \to
[0,1]$ given by
\[
\mu*(A) = \mu(A^{-1}) \textrm{ for all $A \in \mathcal{P}(G)$}
.\] 
Notice that $\mu*$ and $\mu \mapsto \mu*$ are involutions of $\mathcal{PM}(G)$.
Additional for $x \in \ell^{\infty}(G)$ define $x^{*} \in
\ell^{\infty}(G)^{*}$ by
\[
x^{*}(g) = x(g^{-1}) \textrm{ for all $g \in G$}
.\] 
The map $x \mapsto x^{*}$ is an isometric involution of $\ell^{\infty}(G)$. By
duality it gives the isometric involution $u \mapsto u^{*}$ of
$\ell^{\infty}(G)^{*}$ by
\[
u^{*}(x) = u(x^{*}) \textrm{ for all $x \in \ell^{\infty}(G)$}
.\] 

There are some playing nice conditions that happen with all of these operators
namely the following proposition

\begin{propn}
  Let $g \in G, x \in \ell^{\infty}(G), \mu \in \mathcal{PM}(G), u \in
  \ell^{\infty}(G)^{*}$. Then we have that 
  \begin{enumerate}
    \item $(gx)^{*} = x^{*}g^{-1}$
    \item $(xg)^{*} = g^{-1}x^{*}$ 
    \item \ldots and other. See page 84 in the textbook
  \end{enumerate}
\end{propn}

\subsection{Definition of Amenability}%
\label{sub:Definition of Amenability}

At last, we will attempt to define what it means to be an amenable group!
However, we will need some more definitions first. Namely we will need to know
what a left/right invariant probability measure is, and then how these will
interact also with the means that can be placed on the group.

\begin{defn}
  A finitely additive probaility measure $\mu \in \mathcal{PM}(G)$ is left
  invariant if for all $g \in G$ we have that
  \[
  g\mu = \mu
  .\] 
  Similarly for a right invariant measure. A probaility measure $\mu$ is said
  to be bi-invariant if $\mu$ is both left and right invariant.
\end{defn}

We have a similar notion for group means
\begin{defn}
  Suppose that $m \in \mathcal{M}(G)$ then $m$ is left invariant if $m$ is
  fixed under the left group action of $G$ on $\mathcal{M}(G)$. One says that
  $m$ is bi-invariant if $m$ is both-left and right invariant.
\end{defn}

\begin{propn}
  Suppose that $\mu \in \mathcal{PM}(G)$ the $\mu$ is left invariant if and
  only if $\mu^{*}$ is right invariant
\end{propn}

\begin{propn}
  Suppose that $m \in \mathcal{M}(G)$ then $m$ is left invariant if and only if
  $\mu*$ is right invariant
\end{propn}

\begin{propn}
  Let $m \in \mathcal{M}(G)$ then $m$ is left invariant if and only if the
  associated probability measure $\hat{m} \in \mathcal{PM}(G)$ is left-invariant.
\end{propn}
These propositions follow immediately from properties of means and medians
under group actions given in the previous section. Next is potentially the most
important proposition in this chapter, namely what it means for a group to be
amenable.

\begin{propn}
  Let $G$ be a group. Then the following are equivalent.
  \begin{enumerate}
    \item There is a left-invariant finitely additive probability measure on $G$
    \item There is a right-invariant finitely additive probability measure on $G$
    \item There is a bi-invariant finitely additive probability measure on $G$
    \item There is a left-invariant mean on $G$
    \item There is a right-invariant mean on $G$
    \item There is a bi-invariant mean on $G$
  \end{enumerate}
\end{propn}
\begin{proof}
  \textbf{Fill this in later}
\end{proof}

\begin{defn}
  A group $G$ is said to be amenable if it satisfies any of the above
  conditions.
\end{defn}

Some important observations to begin with

\begin{propn}
  Every finite group is amenable
\end{propn}
\begin{proof}
  Suppose that $G$ is a finite group. Then consider the probability measure
  $\mu: \mathcal{P}(G) \to [0, 1]$ given by
  \[
    \mu(A) = \frac{|A|}{|G|}
  .\] 
  This is clearly a bi-invariant probability measure on $G$.
\end{proof}

Next we will present the most important non-amenable group.
\begin{thm}
  The free group on two generators is not amenable.
\end{thm}
\begin{proof}
  \textbf{Fill in later}
\end{proof}

\subsection{Stability Properties}%
\label{sub:Stability Properties}

Oh jeez here we are again. Of course we need to talk about how stable amenable
groups are. It again turns out that they are very stable, and indeed even have
conditions under which we can determine the amenability of a group by looking
at particular normal subgroups.

\begin{propn}
  Every subgroup of an amenable group is amenable
\end{propn}
\begin{proof}
  
\end{proof}

\begin{propn}
  Every quotient of an amenable group is amenable
\end{propn}
\begin{proof}
  
\end{proof}

\begin{propn}
  Let $G$ a gropu and $H$ a normal subgroup of $G$. Suppose that $H$ and
  $G / H$ are amenable. Then the group $G$ is amenable.
\end{propn}
\begin{proof}
  
\end{proof}

\begin{cor}
  The direct product of amenable groups is amenable.
\end{cor}

\begin{propn}
  Every group which is the limit of an inductive system of amenable groups is
  amenable.
\end{propn}
\begin{proof}
  
\end{proof}

\subsection{Solvable Groups}%
\label{sub:Solvable Groups}

\subsection{Folner Conditions}%
\label{sub:Folner Conditions}
The Folner conditions give us another way to classify if a groups is ameanble.
To understand the Foelner conditions we will first show that the following are
equivalent.

\begin{propn}
  Let $G$ be a group, then the following are equivalent
  \begin{enumerate}[(a)]
    \item For all $K \subseteq G$, finite and for all $\epsilon > 0$ there is
      a finite subset $F \subseteq G$ such that
      \[
      \frac{|F \backslash kF|}{|F|} < \epsilon \textrm{ for all $k \in K$}
      .\] 
    \item There is a net $(F_j)_{j \in J}$ of subsets of $G$ such that
      \[
      \lim_{j} \frac{|F \backslash gF|}{|F|} = 0 \textrm{ for all $g \in G$}
      .\] 
  \end{enumerate}
  And similar `right' conditions.
\end{propn}
\begin{proof}
  It is clear that the left and right conditions imply eachother. Now we will
  show that $(a) \iff (b)$. First suppose that $(a)$ holds. Define $J$ as all
  pairs of finite subsets $(K, \epsilon)$, recall that we get this epsilon from
  the assumption of $(a)$. Now we define the partial order $\le$ on $J$ as
  \[
    (K, \epsilon) \le (K', \epsilon') \iff (K \subseteq K' \textrm{ and
    } \epsilon \ge \epsilon'
  .\] 
  J forms a lattice but namely J is a paritally ordered set. For each $j \in J$
  there exists a $F_j \subseteq G$ for which
  \[
      \frac{|F \backslash kF|}{|F|} < \epsilon \textrm{ for all $k \in K$}
  .\] 
  now let $g \in G$ and fix $\epsilon_0 > 0$. Define $j_0 = (\{g\}
  , \epsilon_0$. If $j = (K , \epsilon)$ and $j \ge j_0$ then for some $g \in
  K, \epsilon \le \epsilon_0$ for which
  \[
      \frac{|F \backslash gF|}{|F|} < \epsilon_0 
  .\] 
  Thus the net satisfies $(b)$ as needed. The other direction follows quickly
  by a simlar argument.
\end{proof}

\begin{defn}
  We say that a group $G$ satisfies the Foelner conditions if it satisfies any
  of the above conditions.
\end{defn}




% \subsection{Paradoxical Decompositions}%
% \label{sub:Paradoxical Decompositions}






