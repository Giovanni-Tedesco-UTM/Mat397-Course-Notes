\section{Residually Finite Groups}%
\label{sec:Residually Finite Groups}

\subsection{Residually Finite Groups}
Residually finite groups attempt to generalize some of the nice properties that we 
have about finite groups into groups that are not necessarily finite. We do this by
looking at groups that are almost kinda finite. Namely in this section we will 
explore the properties of Residually Finite groups and Hopfian Groups and will 
later look at some of their dynamical properties.


\begin{defn}
  Let $G$ be a group. We say that $G$ is residually finite if for any element
  $g \in G$ there is a finite group $F$ and a homomorphism $\phi: G \to F$ such
  that $\phi(g) \neq 1_F$.
\end{defn}

\begin{ex}
  All finite groups are residually finite
\end{ex}
\begin{ex}
  $\mathbb{Z}$ is a residually finite group
\end{ex}
\begin{ex}
  The group $GL_{n}(\mathbb{Z})$ is a residaully finite groups.
\end{ex}

Of course we wish to classify such groups and find out exactly when a group is 
residually finite. So we will now try and construct a group which is not residually 
finite

\begin{propn}
  The group $(\mathbb{Q}, +)$ is not residually finite
\end{propn}

To prove this we will need the following definition and lemma

\begin{defn}
  A group $G$ is said to be divisible if for all $g \in G$ and for all $n \in
  \mathbb{Z}$ there is a $h \in G$ such that $g = h^{n}$.
\end{defn}
\begin{lemma}
  Let $G$ be a divisible group and $F$ a finite group. Every homomorphism
  $\phi: G \to F$ is trivial.
\end{lemma}
\begin{proof}
 Let $n = |F|$ and let $g \in G$. Since $G$ is divisible we have that there is
 an $h$ such that $g = h^{n}$. Then we have that
 \[
 \phi(g) = \phi(h^{n}) = \phi(h)^{n} = 1
 .\] 
\end{proof}
Thus every homomorphism is trivial.

Now notice that $(\mathbb{Q}, +)$ is clearly a divisble group. But then we have
as a result there cannot be a finite group $F$ and  $\phi$ such that $\phi(q)
\neq 1_F$. This is indeed our first example of a non-residually finite group.

\begin{propn}
  All non-trivial divisible groups are not residually finite
\end{propn}
Recalling how we proved that $\mathbb{Z}$ is residually finite, we were able to 
look at finite index normal subgroups of $\mathbb{Z}$. Notice that these kinds 
of subgroups  are very useful since we know that we have nice mappings into these 
kinds of finite groups. This leads us to introduce the notion of the residual 
subgroup of a group $G$.

\begin{defn}
  The residual subgroup is the intersection of all finite index subgroups of
  $G$.
\end{defn}
\begin{lemma}
  Let $G$ a group and $H \le G$ then we can define $K = \bigcap_{g \in H}
  gHg^{-1}$. Then we have that $K \trianglelefteq G$. Clearly $K \subseteq H$.
  Moreover, if $H$ is of finite index in $G$ then $K$ of finite index in $G$.
\end{lemma}

\begin{thm}
  Let $G$ a group and $N$ the residual subgroup of $G$. Then
  \begin{enumerate}[(i)]
    \item $N$ is equal to the intersection of all normal subgroups of $G$
    \item $N \trianglelefteq G$
    \item $G$ is residually finite if and only if $N = \{1_G\} $
  \end{enumerate}
\end{thm}
\begin{proof}
\textbf{Fill this in}
\end{proof}

\subsection{Stability Properties}

Now knowing some of the properties of residually finite groups, it is important to 
also explore what we can do to these groups while still preserving their structure. 
The first natural question to ask here is are the subgroups of a residually
finite group residually finite. The answer is yes.

\begin{propn}
  Every subgroup of a residually finite group is residually finite
\end{propn}
\begin{proof}
  \textbf{Fill in Later}
\end{proof}

\begin{propn}
  Suppose that $I$ is an indexing set. Let $(G_i)_{i \in I}$ be a family of
  residually finite groups. $G = \prod_{i \in I}^{} G_i $ is residually finite.
\end{propn}

\begin{cor}
  Direct sums of families of groups are residually finite
\end{cor}
\begin{cor}
  Every finitely generated abelian group is residually finite.
\end{cor}
\begin{cor}
  Let $G$ be a group then the following are equivalent
  \begin{enumerate}[(a)]
    \item G is residually finite
    \item There is a family $F_i$ of finite groups such that $G$ is isomorphic
      to a subgroup of the direct product group $\prod_{i \in I} F_i$
  \end{enumerate}
\end{cor}

\begin{propn}
  If a group $G$ is a limit of projective system of residually finite groups
  $(G_i)_{i \in I}$ then $G$ is residually finite where $G = \varprojlim G_i$.
\end{propn}
\begin{cor}
  Every profinite group is residually finite.
\end{cor}

Next we will state the first of many definitions based on properties of groups. Thes
e will be useful in further classifying groups, and will give us a way to talk about
groups that may not satisfy properties on their own, but may satisfy them in some lo
cal sense.

\begin{defn}
  If $\mathcal{P}$ is a property of gropus then $G$ is virtually $\mathcal{P}$ 
  if $G$ contains a subgroup of finite index which satisfies $\mathcal{P}$.
\end{defn}
\begin{lemma}
  Let $G$ a group and $H \le P$ of finite index. Suppose that $K \le H$ then
  $K$ is a subgroup of finite index of  $G$.
\end{lemma}

\begin{propn}
  Every virtually residually finite group is residually finite.
\end{propn}

\subsection{Hopfian Groups}

\begin{defn}
  A group $G$ is hopfian if every surjective endomorphism $\phi: G\to G$  is
  injective.
\end{defn}

\begin{thm}
  Every finitely generated residaully finite group is Hopfian.
\end{thm}
\begin{lemma}
 Let $G$ a finitely generated group and let $F$ a finite group. Then the set
 $\Hom(G, F)$ is finite. 
\end{lemma}

\begin{proof}[Proof of Theorem 2.3.2]
  
\end{proof}

\subsection{Automorphism Groups of Residually Finite Groups}

First we recall the definition of an automorphism
\begin{defn}
  An automorphism is a bijective endomorphism $\alpha: G \to G$. We define the
  set of all automorphism as $\Aut(G)$. This forms a group under function
  composition and is a subgroup of $\Sym(G)$.
\end{defn}

The main theorem that we try and show in this section is the relationship
between a residually finite group and its automorphism group. Namely we are
going to focus on finitely generated residually finite groups.

\begin{thm}
  Let $G$ a finitely generated residually finite group. Then $\Aut(G)$ is
  residually finite.
\end{thm}

We are going to need the following lemma in order to prove this:
\begin{lemma}
  Suppose that $G$ is a group and $H_1, H_2$ are subgroups of finite index.
  Then the subgroup $H = H_1 \cap H_2$ is a subgroup of finite index.
\end{lemma}
\begin{proof}
  Notice that $g_1 \equiv g_2 \mod(H) \iff g_1 \equiv g_2 \mod(H_1)$ and $g_1
  \equiv g_2 \mod(H_2)$. Therefore there is an injective map from $G / H \to
  G / H_1 \times G / H_2$ given by $gH \mapsto (gH_1, gH_2)$ thus $G / H$ is
  finite.
\end{proof}

\begin{proof}[Proof of theorem 2.4.2]
  \textbf{Type this up later.}
\end{proof}

\subsection{Dynamical Characteristics of Residually Finite Groups}
It is not completely clear why we care about these residually finite groups, and 
indeed it is not clear why they are important to us at given that we are studying 
Cellular Automata and in a broader sense dynamics. However, it turns out that 
residually finite groups indeed have many interesting dynamical properties. These 
are described by the following theorem:

\begin{thm}
  Suppose that $G$ is a group. Then the following are equivalent.
  \begin{enumerate}
    \item G is residually finite
    \item For every set $A$ the set of points $A^{G}$ which have finite $G$ 
      orbit is dense in $A^{G}$.
    \item There is a set $A$ having at least two elements such that the set of
      points which have finite $G$-orbit is dense in $A^{G}$.
    \item There is a Hausdorff topological space $X$ equipped with a continuous
      and faithful action of $G$ such that the set of points of $X$ that have
      finite $G$-orbit are dense in $X$.
  \end{enumerate}
\end{thm}

\begin{lemma}
  Let $G$ a group and $A$ a set having at least 2 elements. Then the action
  of $G$ on $A^{G}$ is faithful.
\end{lemma}
\begin{proof}
  \textbf{Fill in later.}
\end{proof}

\begin{lemma}
  Let $G$ be a residually finite group and let $\Omega$ a finite subset of
  $G$. Then there exists a normal subgroup $K$ of $G$ of finite index, such
  that the restricution of the canonical surjection $\rho: G \to G / K$ to
  $\Omega$ is surjective.
\end{lemma}
\begin{proof}
  \textbf{Fill in later.}
\end{proof}

\begin{proof}[Proof of theorem 2.5.1]
 \textbf{Fill in later.} 
\end{proof}



