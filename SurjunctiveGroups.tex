\section{Surjunctive Groups}%
\label{sec:Surjunctive Groups}

Surjunctive groups are the first example we see of groups related to the behaviour
of the cellular automaton on them. Not only do these groups have interesting 
behaviour in general. They are also a very large subset of all groups. Indeed it is
currently an open problem to show that there are groups that are not surjunctive.

\subsection{Initial Definition}
We first begin by defining what a Surjunctive group is.

\begin{defn}
  Let $G$ be a group and $A$ be a finite set. We say that $G$ is a surjunctive group
  if every injective cellular automata $\tau: A^{G} \to A^{G}$ is surjective.
\end{defn}

\begin{propn}
  Every finite group is Surjunctive.
\end{propn}
\begin{proof}
  If $G$ is a finite group then $A^{G}$ is a finite set and thus every
  surjective cellular automata $\tau: A^{G} \to A^{G}$ is injective.
\end{proof}


\subsection{Stability Properties}
Similar to our exploration of Residually Finite Groups we will explore some of the
stability properties of Surjunctive groups. Additionally we will attempt to come up
with a necessary and sufficient condition for a group to be surjunctive. We first 
notice that:

\begin{propn}
  Every subgroup of a surjunctive group is surjunctive
\end{propn}
\begin{proof}
  \textbf{Fill in later}
\end{proof}

\begin{propn}
  Let $G$ a group then the following are equivalent.
  \begin{enumerate}[(a)]
    \item G is surjunctive
    \item Every finitely generated subgroup of $G$ is surjunctive
  \end{enumerate}
\end{propn}

This now alludes to a relationship between residually finite groups and 
surjunctive groups. This relationship would be quite strong, especially because 
we know that there are already very many residually finite groups and as a result 
it would show us that many groups are surjunctive almost for free.

\begin{thm}
  All Residually Finite Groups are surjunctive
\end{thm}

In order to prove this we will need the following two lemmas.

\begin{lemma}
  Let $G$ a group and $A$ finite. Let $\tau: A^{G} \to A^{G}$ a cellular
  automaton. Then $\tau(A^{G})$ is closed in the prodiscrete topology.
\end{lemma}
\begin{proof}
  Notice that since $A$ finite, then $A^{G}$ is compact, and $\tau: A^{G} \to
  A^{G}$ is continuous. Then we have that $\tau(A^{G})$ is compact and is
  therefore closed since $A^{G}$ is Hausdorff.
\end{proof}

\begin{lemma}
  Let $G$ a group, suppose that $G$ satisfies the following condition. For each
  finite subset $\Omega \subseteq G$ there is a surjunctive group $F$ and a map
  $\phi: G \to F$ such that $\phi|_{\Omega}$ is injective. Then we have that
  $G$ is surjunctive.
\end{lemma}
\begin{proof}
  \textbf{Fill in later}
\end{proof}


\textbf{Marked Groups Section Later}
