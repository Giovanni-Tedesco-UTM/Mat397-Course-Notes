\section{Topology Primer}%
\label{sec:Topology Primer}

\subsection{Uniform Spaces}

Recall from your real analysis class that a function $f: \mathbb{R} \to
\mathbb{R}$ is uniformly continuous if it satisfies the following: For all
$\epsilon > 0$ there is a $\delta > 0$ such that for all $x, y \in \mathbb{R}$.

\[
| x - y| < \delta \implies |f(x) - f(y)| < \epsilon
.\] 
effectively telling us that between any two points of some distance apart, a 
function can only grow so fast. The issues that begin to arise however, are 
when we try and generalize this to a more general topological space. Recall 
that in a general topological space we can define continuity at a point as 
follows. Let $x \in X$ and $V$ a neighbourhood of $f(x)$, we say that
a function $f: X \to Y$ is continuous if there exists a neighbourhood $U$ of
$x$ such that
\[
f(U) \subseteq V
.\] 
Open sets give us an understanding of what points are ``close'' to
another point. They don't however, tell us anything about how close these
points are. So when we try and generalize our definition of uniform continuity
we need to first make sense of what what it means to keep points close to one
another.

Suppose that $X$ is a set. We define the diagonal of $X$ which we will denote
as $\Delta_X$ as 
\[
\Delta_X = \{(x, x): x \in X\} \subseteq X \times X
.\] 
Now consider a set $R \subseteq X \times X$. We can think of this as a set of
elements which satisfy some binary relationship $R$. Notice now we can begin to
talk about all the elements of $X$ which agree with eachother for some
relationship $R$. Additionally suppose that $y \in X$ we can talk about all
points of $X$ which agree with $y$ on $R$ as
\[
  R[y] = \{x \in X: (x, y) \in R\} 
.\] 
Additionally we can denote the inverse of our relationship $R^{-1}$ as
\[
  R^{-1} = \{(x, y) = (y, x) \in R \}
.\] 
naturally if a set $R$ satisfies both $R$ and $R^{-1}$ we say that $R$ is
symmetric. Lastly, suppose that $R$ and $S$ are two sets, we can denote the
composition of these sets as $R \circ S$
\[
R \circ S = \{(x, y): \textrm{ there is a $z$ such that $(x, y) \in R, (z, y)
\in S$}\} 
.\] 
Now by the above we can state the definition of a Uniform Structure.

\begin{defn}
  Suppose that $X$ is a set. A uniform structure on  $X$ is a collection of
  sets $\mathcal{U}$ of subsets $X \times X$ where $\mathcal{U}$ satisfies the
  following:
  \begin{enumerate}
    \item If $V \in \mathcal{U}$ then $\Delta_{X} \in \mathcal{U}$
    \item If $V \in \mathcal{U}, V \subseteq V' \subseteq X \times X$ then $V'
      \in U$
    \item If $V, W \in \mathcal{U}$ then $V \cap U \in \mathcal{U}$
    \item If $V \in \mathcal{U}$ then $V' \in \mathcal{U}$
    \item If $V \in \mathcal{U}$ then there is a $W \in \mathcal{U}$ such that
      $W \circ W \in \mathcal{U}$
  \end{enumerate}
\end{defn}
Intuitively a uniform structure on $X$ is a system of relationships each of
which tell us at most how far each of the points contained inside of it are.
Notice also that every relationship in the set will also contain the diagonal
of $X$.

If a set $X$ is given a uniform structure $\mathcal{U}$ then we call it
a uniform space and we refer to $U \in \mathcal{U}$ as entourages.

\begin{ex}
  Some examples of Uniform spaces are \textbf{fill in later}
\end{ex}

Additionally recall that a basis of $\mathcal{U}$ is a set $\mathcal{B}
\subseteq \mathcal{U}$ if for each $U \in \mathcal{U}$ there is a $V \in
\mathcal{B}$ such that $V \subseteq U$.

\begin{defn}
  Suppose that $X$ is a set and $\mathcal{B}$ a non-empty set of subsets of $X
  \times X$. Then $\mathcal{B}$ a basis for some uniform structure on $X$ if
  and only if
  \begin{enumerate}
    \item If $V \in \mathcal{B}$ then $\Delta_{X} \subseteq V$ 
    \item If $V, W \in \mathcal{B}$ then there is a $U \in \mathcal{B}$ such
      that $U \subseteq V \cap W$
    \item If $V \in \mathcal{B}$ then there is a $W \in \mathcal{B}$ such that
      $W \subseteq V^{-1}$.
    \item If $V \in \mathcal{B}$ then there is a $W \in \mathcal{B}$ such that
      $W \circ W \subseteq V$
  \end{enumerate}
\end{defn}
Consider example B.1.1, notice that each of the sets $V_{\epsilon}$ gives us a way 
to talk about exactly what points $(x,y)$ are epsilon apart from each other by 
simply checking if $(x, y) \in V_{\epsilon}$. This is very powerful as now we have  
way of globally asking questions about what points are close together. Notice 
also that this solves the problem that we were having in our earlier sections of 
how to define uniform continuity. Once we know what our "closeness" is supposed 
to look like, we can create a uniform structure relating points that we consider 
to be "$\epsilon$" close together through the use of defined relationships. This 
naturally leads us to the definition of uniform continuity.

\newpage

\section{Markov-Kakutani Fixed Point Theorem}%
\label{sec:Markov-Kakutani Fixed Point Theorem}

\begin{thm}
  Suppose that $K$ is a non-empty convex, compact subset of a topological
  vectorspace $X$. Let $\mathcal{F}$ be a family of affine maps $f: K \to K$.
  If all elements commute that is for all $f_1, f_2 \in \mathcal{F}, f_1 \circ
  f_2 = f_2 \circ f_1$. Then there exists a point $x \in K$ which is fixed by
  all $f \in \mathcal{F}$.
\end{thm}

To prove this we are going to need the following lemams. 
\begin{lemma}
  Let $K$ be a compact subset of a topologial space $X$ and let $V$ 
  a neighbourhood of 0 in $X$. Then there exists a real number $\alpha > 0$ 
  such that $\lambda K \subseteq V$ for every real number $\lambda$ such that
  $|\lambda| < \alpha$.
\end{lemma}
\begin{proof}
  Since multiplication is continuous we can find for each $x \in X$ an $\alpha_x 
  \in \mathbb{R}$  and a neighbourhood $\Omega_x$ for which
  \[
  |\lambda| < \alpha_x \implies \lambda_x \Omega_x \subseteq V
  .\] 
  These form an open cover of $K$ and thus since $K$ is a compact set we know
  there is a finite subcover $F \subseteq K$ such that
  \[
  K \subseteq \bigcup_{x \in F} \Omega_x
  .\] 
  Now take $\alpha_0 = \min_{x \in F} \{\alpha_x\} $ then we have that
  $\alpha > 0$ and
  \[
  |\lambda| < \alpha \implies \lambda K \subseteq V
  .\] 
\end{proof}

\begin{lemma}
  Let $K$ be a compact subset of $X$. Suppose that $(x_i)$ is a net of points
  and $(\lambda_i) \to 0$ a sequence of real numbers. Then $(\lambda_i x_i)$ 
  converges to 0 in $X$. 
\end{lemma}
\begin{proof}
  Suppose that $V$ is a neighbourhood of 0 in $X$ then there is an $\alpha
  > 0$ such that
  \[
    \lambda K \subseteq V \tag{For all $\lambda < \alpha$}
  .\] 
  Since $\lambda_i \to 0$ there is an $i_0 \in I$ such that for  $i \ge i_0$ we
  have that $|\lambda_i| < \alpha \implies \lambda_i x_i \in V$ for all $i \ge
  i_0$. And so $\lambda_i x_i \to 0$. As needed.
\end{proof}

\begin{lemma}
  Let $K \subseteq X$ a non-empty convex, compact subset of a Hausdorff
  topological vector space $X$ and let $f: K \to K$ an affine continuous map.
  Then  $f$ has a fixed point.
\end{lemma}
\begin{proof}
  Consider the set $C = \{y - f(x): y \in K\}$. Notice that $f$ having a fixed
  point is equivalent to the fact that $0 \in C$. Now let $x \in K$ and
  consider $(x_n)_{n \ge 1}$ defined as
  \[
  x_n = \frac{1}{n} \sum_{k=0}^{n-1} (f^{k}(x) - f^{k + 1}(c))
  .\] 
  We know that each term of the above is in $C$. Notice also that $C$ is
  convext since $K$ is convex and $f$ is an affine function. Therefore, we have
  that $x_n \in C$ for all $n \ge 1$. We see that the above sum is
  a telescoping sum and so we have that
  \[
  x_n = \frac{1}{n}x - \frac{1}{n}f^{n}(x)
  .\] 
  Notice that this must go to 0 by the previous lemmas. Therefore we have
  constructed a sequence of elements of $C$ that converge to 0. Since $C$ is
  compact we have that $C$ is closed and thus $0 \in C$.
\end{proof}
\begin{proof}[Proof of Markov-Kakutani]
  Let $f \in \mathcal{F}$ and consider the set
  \[
  \textrm{Fix}(f) = \{x \in K: f(x) = x\} 
  \] 
  We see that $\textrm{Fix}(f) \neq 0$ by the previous lemma. Suppose now that
  $f, g \in \mathcal{F}$ and let $x \in \textrm{Fix}(f)$ we have that
  \[
  f(g(x)) = g(f(x)) = g(x)
  \] 
  now we can apply the lemma to $g|_{\textrm{Fix}(f)}$ this implies that
  \[
  \textrm{Fix}(f) \cap \textrm{Fix}(g) \neq 0
  \] 
  and so by induction and the finite intersection property we are done.
\end{proof}
